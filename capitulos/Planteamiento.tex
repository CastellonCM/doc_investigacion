\spacing{1.25}
\section{Objetivos}
	\subsection*{Objetivo general}
	Estudiar mediante simulaciones computacionales la distribución lateral de muones en cascadas atmosféricas con energías iniciales en el rango del observatorio HAWC (1-100 TeV).

	\subsection*{Objetivos específicos}	
	\begin{enumerate}
	\item Caracterizar la densidad de muones a varias distancias del eje de la cascada atmosf\'erica en funci\'on de la energ\'ia inicial.
	\item Comparar las distribuciones de muones en cascadas atmosf\'ericas iniciadas por distintas part\'iculas primarias.
	\item Evaluar la influencia del modelo de interacciones hadr\'onicas de altas energ\'ias utilizado en las simulaciones sobre la distribuci\'on lateral de muones.
	\end{enumerate}

\section{Preguntas de investigación}
Al ingresar a la atm\'osfera los rayos c\'osmicos interact\'uan con las part\'iculas de la misma produciendo multiples part\'iculas secundarias, muchas de \'estas pueden detectarse por diversos instrumentos en la superficie terrestre; uno de ellos es el observatorio HAWC, que es capaz de detectar eventos de energ\'ias entre $\sim 1$ TeV y $\sim 100$ TeV. En este trabajo de investigaci\'on se pretende estudiar eventos en dicho rango de energ\'ias. En particular se quiere indagar en las caracter\'isticas de la componente mu\'onica de la distribuci\'on lateral de part\'iculas producidas en cascadas. La componente mu\'onica se ve afectada tanto por la energ\'ia inicial como por la masa del rayo c\'osmico primario, por lo que se requiere precisar esas dependencias. Adicionalmente, debido a las discrepancias que se han reportado entre distintos modelos de interacciones hadr\'onicas es conveniente preguntarse si dichas dependencias son sensibles al modelo utilizado para las simulaciones.

%A partir de datos experimentales del PAO se ha intentado deducir la composición de los rayos cósmicos de las más altas energías, para ello se han realizado ajustes a observables de chubascos atmosféricos tales como la profundidad del máximo y sus fluctuaciones, y el número de muones a nivel del suelo, sin embargo los resultados no son concluyentes. En esta investigación se pretende analizar la relación de la composición con otras observables: la distribución lateral de electrones, la distribución lateral de muones y las densidades de electrones y muones a una distancia fija del eje del chubasco. \\
%
%Para ello se plamtean las siguientes interrogantes: ¿cómo afecta la composición del flujo de UHECR a la distribución lateral de electrones y de muones? Tomando en cuenta una distancia fija a partir del eje del chubasco ¿cómo afecta la composición primaria a las densidades de electrones y muones? ¿Son estos efectos sensibles al modelo de interacciones hadrónicas de altas energías utilizado para las simulaciones?

\section{Justificación}
Debido a su bajo flujo los rayos cósmicos de altas energía se estudian mediante observaciones de las cascadas de partículas que producen en la atmósfera. Las interacciones hadr\'onicas a estas energ\'ias -y por consiguiente las distribuciones de part\'iculas secundarias- no pueden describirse de manera exacta, por lo que se recurre a modelos computacionales que realizan extrapolaciones a partir de datos experimentales a menor energ\'ia. Actualmente existen discrepancias entre las simulaciones realizadas con diferentes modelos de interacciones hadrónicas, así como entre predicciones de simulaciones y datos observacionales. Debido a esto, en este rango de energ\'ias no se ha logrado definir exactamente una \'unica composici\'on de los rayos c\'osmicos. \\

La distribuci\'on lateral de muones contiene informaci\'on sobre la naturaleza de la part\'icula que inici\'o la cascada, por lo que puede usarse en observatorios como HAWC para determinar aspectos como la energ\'ia y la masa del rayo c\'osmico primario. Por otro lado, los muones se producen mayormente en interacciones hadr\'onicas, por lo que la caracterizaci\'on de su flujo a cierta altura es importante para distinguir cascadas iniciadas por rayos gamma de las iniciadas por rayos c\'osmicos. Adem\'as es de inter\'es observar el comportamiento de la densidad de muones en simulaciones con relaci\'on al modelo de las interacciones hadr\'onicas ya que su medici\'on experimental puede ser una herramienta para mejorar los modelos actuales.

%Debido a tales desacuerdos no es posible hacer coincidir simulaciones y observaciones en términos de una única composición de los UHECR.\\
%
%Las magnitudes de un chubasco atmosférico con mayor dependencia de la masa del rayo cósmico primario son la profundidad del máximo y sus fluctuaciones, la profundidad de la máxima producción de muones y el número de muones a nivel del suelo. Tomando en cuenta las mediciones de dichos observables, la colaboración Pierre Auger ha propuesto una composición primaria mixta de los rayos cósmicos. \\
%
%Esta investigación pretende estudiar los efectos de la composición de los UHECR en diferentes observables del desarrollo de chubascos atmosféricos mediante simulaciones, utilizando distintos modelos hadrónicos de altas energías. De esta manera se logrará una mejor comprensión del comportamiento de las distribuciones laterales de electrones y muones y sus densidades en relación a la composición del flujo de rayos cósmicos primarios.\\
%
%Asimismo, se tiene por objetivo analizar la sensibilidad de los resultados de las distribuciones laterales y densidades de partículas al modelo de interacción hadrónica. Dado que la tarea de reconciliar los resultados de las simulaciones ejecutadas con distintos modelos se ha mostrado difícil -debido a que las magnitudes más utilizadas presentan una gran dependencia del modelo-, con este análisis se busca poder sugerir el empleo de estos observables en nuevos estudios acerca de la composición primaria de rayos cósmicos.

\section{Viabilidad}
Para las simulaciones se utilizará el sistema AIRES, éste es de acceso libre y está disponible en línea en el sitio \url{http://aires.fisica.unlp.edu.ar/}, así como toda su documentación; éste ya ha sido instalado en una computadora personal (procesador \textit{AMD Ryzen 5} con seis n\'ucleos y doce hilos, disco de estado s\'olido de 256 GB y 8 GB de memoria RAM) junto con los modelos hadrónicos disponibles, adicionalmente se han realizado ejecuciones de prueba para estimar el tiempo de simulación. Se estima que por cada modelo se empleen tres semanas para todas las simulaciones necesarias, haciendo un total de nueve semanas, siendo éste tiempo razonable dado el tiempo disponible para el desarrollo del proyecto. El an\'alisis de datos se realizar\'a tambi\'en con herramientas de acceso libre, por lo que no ser\'a necesario incurrir a gastos adicionales. Por otro lado cabe mencionar que el asesor principal del proyecto, PhD. Hermes Le\'on Vargas, es investigador en el \'area de astropart\'iculas y parte de la colaboraci\'on del observatorio HAWC, de manera que su experiencia en esta \'area de investigaci\'on es la oportuna para la realizaci\'on del presente proyecto. \\ 

\singlespace
