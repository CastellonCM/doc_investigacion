%%%%%Historia/definiciones
\spacing{1.25}
Los rayos cósmicos -descubiertos por el austríaco Victor Hess en 1914- son partículas cargadas provenientes del exterior de la Tierra que llegan a la misma con energías de hasta $10^{20}$ eV. Éstos son en su mayoría protones (90\%) y núcleos de helio (9\%), el resto son electrones, positrones y núcleos más pesados. Cuando los rayos cósmicos entran a la atmósfera terrestre interactúan con los átomos y moléculas de la misma, generando cascadas de partículas secundarias que se conocen como chubascos atmosféricos, éstos son producto de interacciones electromagnéticas y hadrónicas. \\

Desde el descubrimiento de los rayos cósmicos, se han estudiado ampliamente aspectos como su espectro, su composición, su propagación y su origen. Se conoce que el espectro de rayos cósmicos se extiende desde $10^9$ hasta $10^{20}$ siguiendo una ley de potencias, y que a estas ultraaltas energías se observa una supresión del mismo. También se ha concluido que los rayos cósmicos están compuestos mayormente por protones, y que sus fuentes son principalmente de origen galáctico. Sin embargo las interrogantes fundamentales de la física de rayos cósmicos, relacionadas con su origen y el mecanismo de aceleración con el que alcanzan las energías observadas, siguen vigentes. \\

%%%%%Problema (sim vs obs) (composition problem)
En particular el rango de las ultraaltas energías presenta grandes desafíos para la investigación en esta área. Esto debido principalmente a limitaciones experimentales relacionadas con el bajo flujo de partículas a estas energías, así como limitaciones teóricas en la descripción de las interacciones hadrónicas, ya que los modelos utilizados son fenomenológicos y dependen de los avances en aceleradores de partículas. Actualmente uno de los problemas esenciales es la determinación de la composición de los rayos cósmicos de ultraaltas energías. El problema de la composición consiste en dar una correcta interpretación a los datos observacionales basada en simulaciones de chubascos. \\

Se han medido con bastante precisión diversas propiedades de los chubascos producidos por rayos cósmicos de ultraalta energía que son especialmente sensibles a la composición primaria, no obstante no se ha logrado una interpretación coherente de dichas mediciones en términos de una única composición. Existen importantes discrepancias entre las simulaciones con distintos modelos de interacción hadrónica, de manera que dirigen a conclusiones no compatibles entre sí, y además no compatibles con las observaciones realizadas de las magnitudes dependientes de la composición.\\

%%%%%Objetivo del trabajo 
Uno de los objetivos de este trabajo de investigación es verificar el efecto de una composición mixta de los rayos cósmicos de ultraalta energía en la profundidad del máximo $X_{\text{max}}$ que, junto con otras observables, ha sido ampliamente estudiada con el fin de dilucidar la composición primaria. Asimismo, se tiene por objetivo estudiar el efecto de la composición primaria en otras observables, como lo son las distribuciones laterales de electrones y muones, y su densidad a una cierta distancia. Estos objetivos pretendieron lograrse realizando simulaciones de chubascos atmosféricos tanto de composición puramente ligera o pesada como de composición mixta, comparando los resultados con datos experimentales publicados por la colaboración Pierre Auger, y comparando los resultados obtenidos de simulaciones realizadas con diferentes modelos de interacciones hadrónicas de altas energías.\\
 
%%%%%¿Cómo se hizo? 
Se realizaron tres grupos de simulaciones de ~2400 eventos de chubascos atmosféricos cada uno; el primer grupo fueron chubascos producidos por protones, el segundo por núcleos de hierro, y el tercero por combinaciones de protones, núcleos de helio, de nitrógeno y de hierro, todos con energías iniciales dentro del intervalo $10^{17} \leq E \leq 10^{20}$ eV. El \textit{software} utilizado para simular los chubascos fue AIRES, que cuenta con los modelos hadrónicos de altas energías utilizados (Sibyll 2.3c, EPOS-LHC y QGSJETII-04) como paquetes externos, además de una librería de módulos escritos en Fortran y C++ para realizar los cálculos de las características de los eventos simulados. Adicionalmente, los datos observacionales fueron tomados del repositorio que se encuentra en sitio web del Observatorio Pierre Auger.\\

%%%%%Capítulos
En el capítulo 1 se resume el fundamento teórico necesario para la comprensión física de los chubascos atmosféricos producidos por rayos cósmicos y sus propiedades principales, así como el estado del conocimiento del tema. El capítulo 2 describe el programa AIRES utilizado para las simulaciones, así como las condiciones que se asumen para las mismas. En el capítulo 3 se presentan los resultados obtenidos de las simulaciones, comparando las predicciones para $X_{\text{max}}$ con datos experimentales y observando las dependencias de los resultados de densidades de partículas con el modelo hadrónico y con la composición primaria.




\singlespace