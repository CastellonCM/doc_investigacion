%%%%%Historia/definiciones
\spacing{1.25}
Los rayos cósmicos -descubiertos por el austríaco Victor Hess en 1914- son partículas cargadas provenientes del exterior de la Tierra que llegan a la misma con energías de hasta $10^{20}$ eV. En una primera aproximaci\'on el espectro de los rayos c\'osmicos est\'a compuesto por protones (90\%) y núcleos de helio (9\%), siendo el resto electrones, positrones y núcleos más pesados. Al ingresar a la atmósfera terrestre los rayos c\'osmicos interactúan con los átomos y moléculas que la conforman, generando m\'ultiples partículas secundarias cuyo conjunto se conoce como cascada atmosf\'erica, éstas son producto de interacciones electromagnéticas y hadrónicas. \\

Desde el descubrimiento de los rayos cósmicos, se han estudiado ampliamente aspectos como su espectro, su composición, su propagación y su origen. Se conoce que el espectro de rayos cósmicos se extiende desde $10^9$ hasta $10^{20}$ siguiendo una ley de potencias, y que a estas ultraaltas energías se observa una supresión del mismo. También se ha concluido que de forma aproximada los rayos cósmicos están compuestos mayormente por protones y que sus fuentes son principalmente de origen galáctico. Sin embargo las interrogantes fundamentales de la física de rayos cósmicos, relacionadas con su origen y el mecanismo de aceleración con el que alcanzan las energías observadas, siguen vigentes. \\

%%%%%Problema (sim vs obs) (composition problem)
En particular, el rango de las altas energías presenta grandes desafíos para la investigación en esta área. Esto debido principalmente a limitaciones experimentales relacionadas con el bajo flujo de partículas a estas energías, así como limitaciones teóricas en la descripción de las interacciones hadrónicas, ya que los modelos utilizados son fenomenológicos y dependen de los avances en aceleradores de partículas. Actualmente uno de los problemas esenciales es la determinación de la composición de los rayos cósmicos de altas energías; el problema de la composición consiste en dar una correcta interpretación a los datos observacionales basada en simulaciones de cascadas atmosféricas. \\

En variados observatorios alrededor del mundo se han medido con bastante precisión diversas propiedades de las cascadas producidas por rayos cósmicos de altas energías, que son especialmente sensibles a la energ\'ia y composición primaria. No obstante, no se ha logrado una interpretación coherente de dichas mediciones en términos de una única composición. Existen importantes discrepancias entre las simulaciones con distintos modelos de interacción hadrónica, de manera que dirigen a conclusiones no compatibles entre sí, y además no compatibles con las observaciones realizadas de las magnitudes dependientes de la composición. \\

%%%%%Objetivo del trabajo 
El objetivo de este trabajo de investigaci\'on es estudiar las cascadas atmosf\'ericas a trav\'es de las distribuciones laterales de la componente mu\'onica de las mismas, que debido a que los muones tienen su origen principalmente en interacciones hadr\'onicas, esta componente est\'a estrechamente relacionada con las caracter\'isticas del rayo c\'osmico inicial. Particularmente, se pretende caracterizar dicha distribuci\'on lateral en funci\'on de la energ\'ia primaria y comparar las distribuciones en cascadas iniciadas por n\'ucleos de distinta masa. Lo anterior se llevar\'a a cabo realizando simulaciones de cascadas con diferentes modelos de interacciones hadrónicas, en este caso: Sibyll 2.3d, EPOS-LHC y QGSJETII-04.\\

%%%%%¿Cómo se hizo? 
Se realizar\'an seis grupos de simulaciones de aproximadamente 20,000 eventos de cascadas atmosf\'ericas cada uno; por una parte se simular\'an cascadas producidas por protones, y por otra, cascadas iniciadas por n\'ucleos de hierro, todas ellas haciendo uso de los tres modelos hadr\'onicos separadamente, a manera de poder contrastar los resultados de cada uno. La energ\'ia inicial de las cascadas estar\'a en el rango de 1-100 TeV. El \textit{software} a utilizar para las simulaciones es AIRES, un programa para simular cascadas de manera realista y para manejar la informaci\'on de las mismas. \\

%%%%%Capítulos
El contenido del documento est\'a organizado de la siguiente manera: en el capítulo 1 se resume el fundamento teórico necesario para la comprensión física de las cascadas atmosfericas producidas por rayos cósmicos y sus principales propiedades, así como el estado del conocimiento del tema. En el capítulo 2 se presenta el planteamiento del problema de la investigación incluyendo los objetivos, la justificación y la viabilidad del estudio. En el capítulo 3 se describe el programa AIRES con el que se ejecutar\'an las simulaciones, así como las condiciones que se impondr\'an para las mismas. Por último, el cronograma de actividades de la investigación se muestra en el capítulo 4.




\singlespace