
Se presenta el cronograma de actividades a seguir. A continuación se describen las tareas a realizar en el desarrollo de la investigación:
\begin{center}
\begin{footnotesize}
\begingroup
\setlength{\tabcolsep}{8pt} % Default value: 6pt
\renewcommand{\arraystretch}{1.3} % Default value: 1
\begin{tabular}{|c|c|c|c|c|c|c|c|c|c|c|c|c|c|c|c|}
\hline
\textbf{Semana} & \textbf{01} & \textbf{02} & \textbf{03} & \textbf{04} & \textbf{05} & \textbf{06} & \textbf{07} & \textbf{08} & \textbf{09} & \textbf{10} & \textbf{11} & \textbf{12} & \textbf{13} & \textbf{14} & \textbf{15} \\ \hline
Tarea 1 & \cellcolor{gray}	&	&	&	&	&	&	&	&	&	&	&	&	&	& 	\\ \hline
Tarea 2 &	& \cellcolor{gray} & \cellcolor{gray} & \cellcolor{gray} &	&	&	&	&	&	&	&	&	&	&	\\ \hline
Tarea 3 &	&	&	& 	& \cellcolor{gray}	&	&	&	&	&	&	&	&	&	&	\\ \hline
Tarea 4 &	&	&	&	&	& \cellcolor{gray}	&	&	&	&	&	&	&	&	&	\\ \hline
Tarea 5 &	&	&	&	& \cellcolor{gray}	& \cellcolor{gray}	& \cellcolor{gray}	&	&	&	&	&	&	&	&	\\ \hline
Tarea 6 &	&	&	&	&	&	&	& \cellcolor{gray}	& 	&	&	&	&	&	&	\\ \hline
Tarea 7 &	&	&	&	&	&	&	&	& \cellcolor{gray}	& 	&	&	&	&	&	\\ \hline
Tarea 8 &	&	&	&	&	&	&	& \cellcolor{gray}	& \cellcolor{gray}	& \cellcolor{gray}	& 	&	&	&	&	\\ \hline
Tarea 9 & 	&	&	&	&	&	&	&	&	&	& \cellcolor{gray}	&  	&	&	& 	\\ \hline
Tarea 10 & 	&	&	&	&	&	&	&	&	&	&	& \cellcolor{gray}	&	&	& 	\\ \hline
Tarea 11 & 	&	&	&	&	&	&	&	&	&	&	& \cellcolor{gray}	& \cellcolor{gray}	&	& 	\\ \hline
Tarea 12 & 	&	&	&	&	&	&	&	&	&	&	&	& \cellcolor{gray}	& \cellcolor{gray}	& 	\\ \hline
Tarea 13 & 	&	&	&	&	&	&	&	&	& \cellcolor{gray}	& \cellcolor{gray}	& \cellcolor{gray}	& \cellcolor{gray}	& \cellcolor{gray}	& \cellcolor{gray} 	\\ \hline
\end{tabular}
\endgroup
\end{footnotesize}
\end{center}

\begin{itemize}
\item \textbf{Tarea 01}: redactar los archivos de entrada para cada grupo de simulaciones.	

\item \textbf{Tarea 02}: simular chubascos producidos por protones, hierro y mezcla con el modelo Sibyll 2.3c.

\item \textbf{Tarea 03}: validar las simulaciones de Sibyll reproduciendo resultados de $X_{\text{max}}$.

\item \textbf{Tarea 04}: analizar los resultados de las distribuciones laterales obtenidas con Sibyll.

\item \textbf{Tarea 05}: simular chubascos producidos por protones, hierro y mezcla con el modelo EPOS-LHC.

\item \textbf{Tarea 06}: validar las simulaciones de EPOS-LHC reproduciendo resultados de $X_{\text{max}}$.

\item \textbf{Tarea 07}: analizar los resultados de las distribuciones laterales obtenidas con EPOS-LHC.

\item \textbf{Tarea 08}: simular chubascos producidos por protones, hierro y mezcla con el modelo QGSJETII-04.

\item \textbf{Tarea 09}: validar las simulaciones de QGSJETII-04 reproduciendo resultados de $X_{\text{max}}$.

\item \textbf{Tarea 10}: analizar los resultados de las distribuciones laterales obtenidas con QGSJETII-04.

\item \textbf{Tarea 11}: comparar resultados de los tres modelos.

\item \textbf{Tarea 12}: comparar los resultados de las simulaciones con composición mixta con datos experimentales.

\item \textbf{Tarea 13}: redactar informe final.
\end{itemize}



