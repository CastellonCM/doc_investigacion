\spacing{1.5}


Uno de los principales problemas estudiados en la física de rayos cósmicos es el de la composición primaria de los rayos cósmicos que entran a la atmósfera terrestre con ultraaltas energías ($>10^{17}$ eV). La colaboración Pierre Auger propone la hipótesis de una composición mixta que contiene fracciones de elementos con masas que se encuentran entre la del hidrógeno y la del hierro. Se ha verificado el impacto de la composición mixta propuesta sobre observables de los chubascos atmosféricos, como lo son la profundidad del máximo y su desviación estándar, realizando simulaciones de dichos eventos con el programa AIRES utilizando tres modelos distintos para describir las interacciones hadrónicas que producen el desarrollo de los chubascos. Se han comparado los resultados de las simulaciones con los datos obtenidos por el Observatorio Pierre Auger, concluyendo que se apoya la hipótesis de la composición mixta, pero que, basándose en los resultados de la profundidad del máximo, es pertinente proponer que dicha composición sea más ligera, particularmente en las energías más altas del espectro de ultraaltas energías.


\singlespacing