\spacing{1.25}

A partir de la hipótesis de la colaboración Pierre Auger, que sugiere que los rayos cósmicos primarios que entran a la atmósfera terrestre con ultraaltas energías tienen una composición mixta conformada por núcleos de hidrógeno, helio, nitrógeno y hierro, se han realizado simulaciones de chubascos iniciados por rayos cósmicos de dicha naturaleza, calculando magnitudes de los mismos que son sensibles a la masa de la partícula inicial: $X_{max}$ y $\sigma X_{max}$. La hipótesis de la composición mixta es apoyada por los resultados de las simulaciones, ya que para ambas observables éstos se encuentran entre los datos que se obtendrían considerando una composición únicamente ligera o una composición únicamente pesada. \\

El modelo EPOS-LHC el que más acertadamente reproduce los datos observacionales de $X_{max}$ para todo el rango de las ultraaltas energías, no obstante incluso este modelo tiende a subestimar la profundidad de máximo para las energías más altas. Los otros dos modelos subestiman esta cantidad en intervalo aun más amplio. Dicha discrepancia con los datos sugieren una composición de los UHECR primarios más ligera que la propuesta, en particular para últimas energías del espectro. \\

Pese a que los resultados de las fluctuaciones $\sigma X_{max}$ también apoyan la hipótesis de la composición mixta, los actuales resultados para esta magnitud no se acoplan de buena manera a los datos obtenidos por el Pierre Auger. En este caso, el modelo que ha reproducido mejor la tendencia que siguen los datos es QGSJETII-04, sin embargo los tres modelos tienen un error considerablemente alto debido al que el número de eventos simulados a comparación con el número de eventos reales medidos es distinto, lo que puede estar incidiendo en el cálculo de las fluctuaciones del máximo chubasco a chubasco. Por tanto, aunque de igual manera los resultados se encuentran entre una composición ligera y una totalmente pesada, no puede dilucidarse a través de ellos algún cambio evidente en la composición mixta primaria con la que se llevaron a cabo las simulaciones. \\

Cabe resaltar, finalmente, que la reproducción de los datos experimentales por medio de simulaciones computacionales en función de una única composición primaria mixta es altamente dependiente del modelo de interacción hadrónica utilizado. Lo anterior se evidencia en las diferentes proporciones de partículas en la composición propuestas para cada uno de los modelos utilizados, así como en las claras discrepancias entre los resultados de las simulaciones con los tres modelos para las dos observables consideradas. \\

Se propone como trabajo a futuro verificar el efecto de la composición primaria mixta en otras propiedades de los chubascos, como la producción de partículas secundarias (fotones, electrones o muones) y su distribución lateral e igualmente su desarrollo longitudinal, particularmente su profundidad de máxima producción. Se recomienda que se simule un mayor número de eventos que sea proporcional al espectro de UHECR medido, invirtiendo más recursos computacionales de los que se disponen actualmente.

\singlespacing