\spacing{1.5}


Uno de los principales problemas estudiados en la física de rayos cósmicos es el de la composición primaria de los rayos cósmicos que entran a la atmósfera terrestre con ultraaltas energías ($>10^{17}$ eV). La colaboración Pierre Auger propone una composición mixta que contiene fracciones de elementos con masas que se encuentran entre la del hidrógeno y la del hierro. Se ha verificado el impacto de la composición mixta propuesta sobre observables de los chubascos atmosféricos, como la profundidad del máximo. También se ha estudiado el efecto de la composición primaria en las distribuciones laterales de electrones y muones, así como en su densidad a cierta distancia. Lo anterior se ha hecho realizando simulaciones de dichos eventos con el programa AIRES utilizando tres modelos distintos de interacciones hadrónicas de altas energías. Se han comparado los resultados de profundidad del máximo con datos del Observatorio Pierre Auger, apoyando la hipótesis de una composición mixta. Se observó en la mayoría de casos que el efecto de la masa de la partícula primaria en las distribuciones laterales es evidente a cortas distancias del eje del chubasco pero las diferencias disminuyen a medida la distancia crece. Similarmente se aprecia que al aumentar la energía primaria, las distribuciones laterales de muones se vuelven indistinguibles; ajustes a una función de tipo NKG ilustran estas dependencias así como las diferencias entre los modelos, de los cuales QGSJETII-04 es el que más discrepa de los otros. Por otro lado, los resultados de densidad de electrones a $r_{opt}$ muestran ser independientes de la composición primaria, mientras que la de muones sí muestra una pequeña dependencia; ambas siendo independientes del modelo hadrónico. 

 
\singlespacing