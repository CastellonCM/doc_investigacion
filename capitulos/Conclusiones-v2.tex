\spacing{1.25}
%\vspace{-5mm}
En este trabajo se estudi\'o la distribuci\'on lateral de la componente mu\'onica de cascadas atmosf\'ericas iniciadas por rayos c\'osmicos con energ\'ia entre 1 y 100 TeV. Se simularon cascadas con el programa AIRES, a partir de dichas simulaciones se estudi\'o la distribuci\'on lateral de muones en la ubicaci\'on del observatorio HAWC (4100 m s.n.m): ajustando los datos a una funci\'on de tipo NKG, utilizada en HAWC para describir la distribuci\'on lateral de la carga efectiva \cite{Malone2018}, se caracteriz\'o la componente mu\'onica y su evoluci\'on con la energ\'ia primaria. Se analiz\'o el efecto de la part\'icula primaria, comparando protones y n\'ucleos de hierro; del \'angulo zenital de incidencia, comparando cascadas verticales ($0<\theta<20^{\circ}$) e inclinadas ($20<\theta<45^{\circ}$); y se compararon resultados de diferentes modelos de interacciones hadr\'onicas de altas energ\'ias. Adem\'as se estim\'o el n\'umero de muones que podr\'ian detectarse en HAWC.\\

Las distribuciones laterales de muones de cascadas de protones simuladas con AIRES coinciden razonablemente con resultados obtenidos con el programa CORSIKA considerando las mismas energ\'ias primarias en la misma ubicaci\'on \cite{Parsons2019}. Si bien se observan discrepancias, \'estas se explican por los diferentes \'angulos de incidencia de las cascadas simuladas. En general la densidad de muones tiene valores entre aproximadamente $10^{-6}$ (cascadas de hierro de 1 TeV) y $10^{-2}$ (cascadas de 100 TeV) part\'iculas por metro cuadrado; la densidad incrementa a medida que aumenta la energ\'ia primaria. A pesar de que en una primera aproximaci\'on se esperar\'ia que las distribuciones de muones en cascadas de hierro estuviesen por encima de las de protones, en las distancias consideradas ($R<300$ m) se observa lo contrario para las energ\'ias m\'as bajas. \\

Por su parte, los par\'ametros del ajuste, $A$ y $s$, en funci\'on de la energ\'ia primaria muestran comportamientos opuestos: $A$ incrementa con la energ\'ia mientras que $s$ disminuye. El par\'ametro $s$ calculado para simulaciones de diferentes modelos de interacciones hadr\'onicas no muestra diferencias significativas entre ellas, siendo la mayor de un 10\%, mientras que $A$ presenta diferencias de hasta 68\% en cascadas de hierro (entre QGSJET y Sibyll), sin embargo hay poca variaci\'on del par\'ametro en todo el intervalo de energ\'ia (valores entre 0 y 0.008), por lo que en ninguno de los casos puede asegurarse una dependencia clara del modelo. Al comparar los par\'ametros obtenidos con diferente part\'icula primaria se confirma que hasta una energ\'ia de $\approx 40$ TeV las cascadas de protones muestran m\'as muones en la regi\'on considerada, y adem\'as indican menor edad de las cascadas en todo el intervalo de energ\'ia. Igualmente, al separar los resultados entre cascadas verticales e inclinadas, se observan mayores valores de $A$ en las verticales, a diferencia de $s$, cuyos valores son mayores en las inclinadas. Estos efectos cobran m\'as importancia a partir de los 10 TeV. \\

Por \'ultimo, la estimaci\'on del n\'umero de muones en HAWC sugiere que no ser\'ia posible detectar cascadas con energ\'ias menores a $\approx 5$ TeV ya que en un radio de 70 m no se observan suficientes muones. El n\'umero de muones aumenta gradualmente hasta que para una energ\'ia primaria de 100 TeV podr\'ian detectarse entre 80 y 100 muones (dependiendo del modelo utilizado). Los ajustes a la funci\'on de tipo NKG subestiman $N_{\mu}$ en un promedio del 10\% con respecto a los valores extra\'idos de la simulaci\'on. \\

Cabe mencionar que este trabajo es un primer intento de realizar predicciones para el observatorio HAWC con simulaciones del programa AIRES, siendo que \'este se utiliza usualmente para simulaciones de ultraaltas energ\'ias. Adem\'as, si bien el n\'umero de muones en una cascada es com\'unmente utilizado para estimar las caracter\'isticas del rayo c\'osmico primario, aqu\'i se presenta la posibilidad de utilizar la forma de su distribuci\'on lateral con el mismo fin, mostrando las diferencias de los par\'ametros de la funci\'on que la describe seg\'un su energ\'ia, masa y \'angulo de zenital de incidencia; al mismo tiempo, se eval\'uan dichos par\'ametros como observables que permitan poner a prueba la capacidad de los modelos de interacciones hadr\'onicas para reproducir los datos experimentales de HAWC.

\subsection*{Trabajo a futuro}
Como trabajo a futuro se sugiere realizar simulaciones para m\'as subintervalos de energ\'ia entre 1 y 100 TeV a fin de tener m\'as estad\'istica para calcular los par\'ametros del ajuste. Para tener una mejor idea de la evoluci\'on de la distribuci\'on lateral de muones con la part\'icula primaria, es necesario simular, adem\'as de cascadas de protones y hierro, cascadas iniciadas por rayos c\'osmicos de masas intermedias. Tambi\'en con el objetivo de realizar una comparaci\'on m\'as exacta entre resultados de AIRES y de CORSIKA y asegurarse de que son congruentes entre s\'i, es conveniente reproducir las simulaciones de otros autores a cabalidad, considerando los mismos valores de energ\'ias primarias, \'angulo de incidencia y energ\'ia umbral de los muones. Adem\'as, se contempla buscar otra funci\'on $\rho_{\mu}(r)$ que describa de mejor manera las distribuciones laterales de muones que resultan de las simulaciones. 

\singlespacing

