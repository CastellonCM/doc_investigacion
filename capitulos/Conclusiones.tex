\spacing{1.25}

A partir de la hipótesis de la colaboración Pierre Auger, que sugiere que los rayos cósmicos primarios que entran a la atmósfera terrestre con ultraaltas energías tienen una composición mixta conformada por núcleos de hidrógeno, helio, nitrógeno y hierro, se han realizado simulaciones de chubascos iniciados por rayos cósmicos de dicha naturaleza, calculando magnitudes de los mismos que son sensibles a la masa de la partícula inicial: $X_{\text{max}}$, distribuciones laterales de electrones y muones, y sus densidades a una distancia de 1000 m del eje del chubasco. Reproduciendo resultados ya bien conocidos, la hipótesis de la composición mixta es apoyada por los resultados de $X_{\text{max}}$. \\

El modelo EPOS-LHC el que más acertadamente reproduce los datos observacionales de $X_{max}$ para todo el rango de las ultraaltas energías, no obstante incluso este modelo tiende a subestimar la profundidad de máximo para las energías más altas. Los otros dos modelos subestiman esta cantidad en un intervalo aun más amplio, siendo el modelo QGSJETII-04 el que predice menores profundidades debido a su baja multiplicidad en las interacciones hadrónicas con respecto a los otros dos modelos. \\

En el caso de las distribuciones laterales resultantes de chubascos de todo el intervalo de energías, los tres modelos producen formas similares pero QGSJETII-04 muestra menor densidad de partículas a nivel de suelo, sin embargo este efecto no es tan drástico cuando se consideran las distribuciones laterales de chubascos en un pequeño subintervalo de energía primaria. También se observa que al aumentar la energía la forma de las distribuciones se mantiene similar, aumentando en general el número de partículas. \\

El efecto de la composición primaria es contrario para electrones y muones, encontrando que los chubascos producidos por protones muestran más electrones, mientras los producidos por hierro muestran mayor cantidad de muones a nivel de suelo; no obstante, a energías primarias mayores las distribuciones de distintas composiciones casi se sobrelapan, por lo que se concluye que a las más altas energías del rango considerado, las distribuciones laterales de electrones y muones no son sensibles a la masa  de la partícula que inició el chubasco. \\

Por último, las densidades de partículas a la distancia óptima del PAO en función de la energía inicial no muestran una clara dependencia del modelo de interacciones hadrónicas, a pesar de que las razones entre modelos son más fluctuantes. La densidad de electrones no presenta dependencia tampoco de la composición primaria, mientras que la densidad de muones aumenta sus diferencias entre composiciones al aumentar la energía; las diferencias causadas por la composición se ilustran con los valores obtenidos del ajuste de los datos a una ley de potencias. \\

Cabe resaltar, finalmente, que la reproducción de datos experimentales de profundidad del máximo por medio de simulaciones computacionales en función de una única composición primaria mixta es altamente dependiente del modelo de interacción hadrónica utilizado. Lo anterior se evidencia en las diferentes proporciones de partículas en la composición propuestas para cada uno de los modelos, así como en las claras discrepancias entre los resultados de las simulaciones. Caso contrario, las distribuciones laterales y densidad de partículas a energías iniciales determinadas no tienen una dependencia significativa con los modelos, obteniendo resultados similares con los tres. Por lo que, a pesar de que el efecto de la composición primaria es menos notorio, estas magnitudes parecen prometedoras para investigar la composición primaria de los rayos cósmicos de ultraaltas energías mediante comparaciones entre simulaciones y observaciones. \\

Se propone como trabajo a futuro verificar el efecto de la composición primaria mixta en otras propiedades de los chubascos, como la producción de partículas secundarias (fotones, electrones o muones) y su distribución lateral e igualmente su desarrollo longitudinal, particularmente su profundidad de máxima producción, así como realizar comparaciones detalladas entre datos de detección de partículas producidas en chubascos y simulaciones. Se recomienda que se simule un mayor número de eventos que sea proporcional al espectro de UHECR medido, invirtiendo más recursos computacionales de los que se disponen actualmente.

\singlespacing