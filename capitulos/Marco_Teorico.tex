\spacing{1.25}
\section{Rayos Cósmicos}
Los rayos cósmicos (RC) son partículas cargadas aceleradas a altas energías que se propagan por el espacio y llegan a la atmósfera terrestre. En una primera aproximaci\'on de todo el espectro de RC un $90\%$ de estas partículas son protones, un $9\%$ núcleos de helio y el resto son electrones, núcleos más pesados y antipartículas. La mayoría de RC son relativistas; su espectro de energías está entre $\sim 10^9$ y $\sim 10^{20}$ eV, y sigue una ley de potencias. Actualmente se tiene conocimiento de fuentes de RC de origen galáctico y extragaláctico \cite{Gaisser1990}. A continuación se describen algunos aspectos del desarrollo histórico de la investigación sobre los RC.
	
	\subsection{Descubrimiento y naturaleza de los RC}
	En el año 1900 se realizaban experimentos para estudiar la ionización causada por elementos radiactivos, en estos se observó que el aire contenía algún tipo de radiación que también era capaz de ionizar. A partir de esto se inició la búsqueda del origen de dicha radiación. Se repitieron los experimentos a alturas de $300$ a $1300$ m, esperando que si la fuente de la radiación fuese la corteza terrestre, ésta disminuiría con la altura. En 1911-1912, el austriaco Victor Hess realizó experimentos en globo a alturas de hasta $5200$ m, con los que concluyó que la radiación debía originarse fuera de la Tierra y que, comparando mediciones de día y de noche, no provenía del sol. Victor Hess es considerado el descubridor de los rayos cósmicos \cite{Extremas}.\\
	
	Posteriormente se inició la búsqueda de la naturaleza de estas partículas, siendo el candidato más popular los rayos gamma. En 1927 Jacob Clay observó una disminución de la radiación en bajas latitudes. Esto fue explicado en 1932 por Arthur Compton como la acción del campo magnético de la Tierra sobre los RC, llevando a la conclusión de que la mayor parte de las partículas en cuestión debían tener carga eléctrica, y estudiando los efectos geomagnéticos este-oeste se dedujo que casi todas las cargas eran positivas. Finalizando la década de 1940, experimentos de detección directa realizados por Schein establecieron que aproximadamente el $99\%$ de los RC son protones, núcleos de Helio y núcleos más pesados y que sólo el $1\%$ son electrones, positrones y rayos gamma \cite{Dorman2004}.

	\subsection{Producción de RC}		
	En la figura \ref{fig:espectro} se muestra el espectro observado de RC, el cual está bien representado por una ley de potencias:
	\begin{align}
	\dv{N}{E}=E^{-(\gamma+1)}.
	\end{align}	
	El índice $\gamma$ tiene un valor aproximadamente constante de $2.7$ con dos ligeros cambios: uno a $\sim 10^{16}$ eV, conocido como la \textit{rodilla}, y el otro a $\sim 10^{19}$ eV conocido como el \textit{tobillo} \cite{Extremas}. El espectro se extiende desde  $\sim 10^9$ hasta $\sim 10^{20}$ eV.
	
	\begin{figure}[h]
	\centering	
	\includegraphics[width=0.5\textwidth]{Figuras/espectro_RC} 
	\caption{Intensidad del flujo de rayos cósmicos en función de su energía. Éste está bien representado por una ley de potencias $E^{-2.7}$. (Tomada de \cite{Poderosas}).}
	\label{fig:espectro}
	\end{figure}	

	Por tanto, debe haber un mecanismo capaz de acelerar partículas a tales energías y que reproduzca la forma del espectro observado. En 1949, Fermi propuso un mecanismo en el que las partículas podían ganar energía en colisiones estocásticas en regiones del espacio donde existiesen campos magnéticos turbulentos, como las ondas de choque resultado de un colapso gravitacional, por ejemplo. Se considera que una partícula de prueba tiene un incremento de energía proporcional a la misma $\Delta E = \xi E$ en cada colisión, luego de $n$ colisiones la energía de la partícula será \cite{Gaisser1990}
	\begin{align}
	E_n = E_0 \qty(1+\xi)^n,
	\end{align}
donde $E_0$ es la energía con la que entra al proceso de aceleración. Tomando en cuenta la probabilidad $P_{esc}$ de que la partícula escape de la región de aceleracón, la proporción de partículas que se aceleran a energías mayores a un valor $E$ es
	\begin{align}
	N(\geq E) \propto \frac{1}{P_{esc}}\qty(\frac{E}{E_0})^{-\gamma},
	\end{align}
	con $\gamma = -\ln(1-P_{esc})/\ln(1+\xi) \approx P_{esc}/\xi$, de manera que este mecanismo efectivamente reproduce la ley de potencias que caracteriza al espectro de RC. \\
	
	El mecanismo de Fermi se describe en dos situaciones físicas: nubes de plasma magnetizadas (aceleración de Fermi de segundo orden) y frentes de onda de choque (aceleración de Fermi de primer orden). En la aceleración de segundo orden se considera una partícula que entra a la nube con cierta velocidad, donde cambia su dirección de modo aleatorio por interacciones con el campo magnético turbulento en el interior, mediante este proceso se tiene $\xi=(4/3) \beta^2$, donde $\beta= V/c$ es la velocidad de la nube; en la aceleración de primer orden se considera que la partícula atraviesa la onda de choque e interactúa con el campo magnético del gas que éste va dejando detrás (\textit{downstream}), en este caso $\xi=(4/3) \beta$, donde $\beta= V/c$ se refiere la velocidad del gas detrás del choque respecto al gas delante del choque (\textit{upstream}).
	
	\subsection{Fuentes de RC}
	Luego de establecer cómo pueden acelerarse las partículas, se buscaron objetos astronómicos que cumplan las condiciones necesarias para que el proceso se lleve a cabo. Para que el proceso sea eficaz, la partícula debe estar contenida en una región de radio $R$, tal que se cumpla la siguiente relación \cite{DeAngelis2015}:	
	\begin{align}
	E[\text{PeV}] \simeq B[\mu\text{G}]\times R[\text{pc}].
	\end{align}
	Ésta es llamada relación de Hillas, ilustrada en la figura \ref{fig:Hillas}, en la que también pueden observarse los potenciales aceleradores. Como fuentes de RC de origen galáctico pueden mencionarse las estrellas de neutrones de rápida rotación (púlsares) y los remanentes de supernova, mientras que en el caso extragaláctico se consideran los núcleos galácticos activos y los destellos de rayos gamma.
	
	\begin{figure}[h]
	\centering	
	\includegraphics[width=0.5\textwidth]{Figuras/Hillas_Relation} 
	\caption{La gráfica de Hillas representa las potenciales fuentes de rayos cósmicos según la intensidad de su campo magnético y su tamaño. (Tomada de \cite{DeAngelis2015}).}
	\label{fig:Hillas}
	\end{figure}	

	\subsection{Propagación de RC}
	La presencia de campos magnéticos en el espacio limita la posibilidad de estudiar las fuentes de RC a través de ellos. Los RC llegan a la Tierra isotrópicamente; llegan de todas direcciones con la misma frecuencia, lo que sugiere una trayectoria casi aleatoria desde la fuente hacia la Tierra. Dentro de la galaxia las partículas pueden sufrir varios procesos: difusión en campos magnéticos, convección por vientos galácticos, pérdidas o ganancias de energía, colisiones nucleares con gas interestelar y decaimientos. Para describir la propagación de RC debe resolverse la ecuación de transporte \cite{Gaisser1990}:
	\begin{align}
	\pdv{\mathcal{N}}{t} &= \nabla \cdot \qty(D_i \nabla \mathcal{N}_i)-\pdv{E} \qty[b_i(E)\mathcal{N}_i(E)]-\nabla \cdot u \mathcal{N}_i(E) \nonumber \\
	&+ Q_i(E,t) - p_i \mathcal{N}_i + \frac{v \rho}{m} \sum_{k \geq i}\int \frac{\dd \sigma_{i,k}\qty(E,E')}{\dd E}\mathcal{N}_k(E')\dd E',
	\end{align}
	que contempla los procesos mencionados para calcular la densidad de partículas con energías entre $E$ y $E+\dd E$. Los seis términos de la ecuación representan, respectivamente: la difusión, ganancias de energía, convección, la inyección de partículas, pérdida de partículas por colisiones o decaimientos, cascadas de decaimientos o fragmentación nuclear. 

\section{Interacciones de los RC}
	\subsection{Interacciones electromagnéticas}
	Las partículas cargadas en general interactúan con átomos; estas pueden ionizarlos, excitarlos o producir fotones. Para electrones y positrones a altas energías es relevante la radiación de frenado o \textit{bremsstrahlung}, en la cual partículas cargadas emiten radiación al ser deflectadas por el campo eléctrico de los átomos en un medio. En este proceso, la fracción de energía que la partícula pierde puede describirse por \cite{DeAngelis2015}:
	\begin{align}
	\frac{1}{E} \dv{E}{x} \simeq -\frac{1}{X_0},
	\end{align}
	donde $X_0$ es la longitud de radiación que es dependiente del medio.\\
	
	Por otro lado, los fotones interactúan con un medio principalmente mediante efecto fotoeléctrico (emisión de un electrón de un material que ha absorbido un fotón), dispersión de Compton (transferencia de energía de un fotón hacia un electrón mediante una colisión) y producción de pares electrón-positrón. Este último proceso siendo el más relevante a altas energías; al interactuar con el campo eléctrico de un núcleo, el fotón tiene cierta probabilidad de formar un par $e^{-}-e^{+}$, con una longitud de interacción:
	\begin{align}
	\lambda \simeq \frac{9}{7} X_0.
	\end{align}	
	Los fotones también puede sufrir otros procesos como dispersión de Rayleigh, que puede tener importancia para el transporte de la luz a través de la atmósfera, o interacciones fotonucleares (excitación de núcleos) que se dan a energías alrededor de $10$ MeV.
	
	\subsection{Interacciones hadrónicas}
	Los RC primarios están mayoritariamente conformados por hadrones, como lo son los protones y núcleos. Los hadrones se describen mediante el modelo de quarks \cite{DeAngelis2015}, partículas que interactúan mediante la interacción nuclear fuerte y que, según observaciones, no existen de manera aislada sino en estados ligados de dos o tres quarks. Este tipo de modelos se estudian desde la \textit{cromodinámica cuántica} (QCD por sus siglas en inglés), donde se propone el concepto de \textit{color} como la carga que origina las interacciones fuertes, y el de \textit{gluón} como la partículas mediadora. \\
		
	\begin{wrapfigure}{r}{0.3\textwidth}
	\includegraphics[width=0.3\textwidth]{Figuras/string_frag} 
	\caption{De abajo hacia arriba; fragmentación de una cuerda creando un nuevo par de quarks en el campo de color. (Tomada de \cite{DeAngelis2015}).}
	\label{fig:string}
	\end{wrapfigure}		
	
	Para describir las interacciones hadrónicas se necesitan modelos fenomenológicos apoyados en QCD. Un modelo usado comúnmente es el modelo de cuerdas de Lund (o \textit{string model}) \cite{FragModels}; cuando los hadrones interactúan se forma un campo de color (cuerda) entre pares quark-antiquark, la energía potencial en dicha cuerda aumenta hasta fragmentarse y formar otros quarks que a su vez pueden formar hadrones, como se ilustra en la figura \ref{fig:string}. También suele utilizarse el modelo de \textit{minijet}, para tomar en cuenta la multiplicidad de partículas producidas. \\

	Actualmente existen varios generadores Monte Carlo de eventos hadrónicos; describen este tipo de interacciones basándose en diferentes modelos para ciertos aspectos de la interacción y en datos de colisionadores de partículas. Ejemplos de estos son SIBYLL \cite{Sibyll}, QGSJET \cite{qgsjet} y EPOS \cite{EPOS}, que están especializados en interacciones de altas energías. 
	

\section{Cascadas atmosf\'ericas}
	Una cascada (tambi\'en llamada chubasco) atmosf\'erica (español para \textit{Air Shower}) es una cascada de partículas generada por la interacción de un rayo cósmico en la alta atmósfera. Antes de profundizar en cómo se desarrollan estas cascadas, conviene describir brevemente las principales características de la atmósfera.
	
	\subsection{Atmósfera terrestre}
	La capa de aire que rodea la Tierra se extiende hasta una altura  mayor a $100$ km. Según el modelo \textit{US Standard Atmosphere}, la atmósfera está compuesta principalmente por N$_2$ ($78$ \%), O$_2$ ($21$ \%) y Ar ($0.9$ \%). Acorde al mismo modelo, la densidad del aire es función de la altura:
	\begin{align}
	\rho(h)=\rho_0 e^{-\frac{h}{h_a}},
	\end{align}
	donde $\rho_0 = 1.22\times 10^{-3}$ g/cm$^3$ y $h_a = 8.2$ km. Sin embargo, en el estudio de los chubascos atmosféricos es más frecuente utilizar el concepto de \textit{profundidad} en lugar de la altura. La profundidad $X$ indica la cantidad de materia que atraviesa una partícula al moverse de un punto a otro. Esta se relaciona con la densidad mediante:
	\begin{align}
	X = \int_h^{\infty} \rho (h) \dd h.
	\end{align}

	\subsection{Desarrollo de una cascada atmosférica}	
	Las cascadas atmosf\'ericas se desarrollan de forma compleja como una combinación de cascadas electromagnéticas y producción de partículas por interacciones hadrónicas \cite{Matthews2005}. A grandes rasgos, una interacción hadrónica entre el rayo cósmico primario y un núcleo de la atmósfera produce múltiples partículas secundarias: una partícula principal (con la mayor parte de la energía inicial) que puede iniciar otro chubasco, y un gran número de mesones, principalmente piones cargados ($\pi^{\pm}$) y neutros ($\pi^0$).\\

	La componente electromagnética de la cascada es generada por los piones neutros al decaer en fotones ($\pi^0 \rightarrow \gamma \gamma$), esta cascada consiste en producción de pares ($\gamma\rightarrow e^+ e^-$) y bremsstrahlung ($e^{\pm} \rightarrow e^{\pm}\gamma$). Por su parte, los piones cargados pueden volver a interactuar hadrónicamente mientras tengan suficiente energía, luego de eso decaerán en neutrinos y muones ($\pi^{-} \rightarrow \mu^- \bar{\nu}_{\mu}$, $\pi^{+} \rightarrow \mu^+ \nu_{\mu} $). El desarrollo de un chubasco se ilustra en la figura \ref{fig:airshower}.\\
				
	\begin{figure}[h]
	\centering
	\includegraphics[width=0.6\textwidth]{Figuras/air_shower} 
	\caption{Esquema de la formación y desarrollo de un chubasco atmosférico. Se observa la componente hadrónica y la electromagnética. (Tomada de \cite{Poderosas}).}
	\label{fig:airshower}
	\end{figure}	
	
	La propagación de partículas (nucleones en particular) a través de la atmósfera, puede describirse con la ecuación de cascada:
	\begin{align}
	\dv{N(E,X)}{X} = -\frac{N(E,X)}{\lambda_N(E)} + \int_E^{\infty}\frac{N(E',X)}{\lambda_N(E)} F_{NN}(E,E') \frac{\dd E'}{E},
	\end{align}
	donde $N(E,X) \dd E$ es el flujo de nucleones a una profundidad $X$ en la atmósfera con energías entre $E$ y $E+\dd E$, $\lambda_N$ es la longitud de interacción del nucleón en el aire y $F_{NN}$ es la sección eficaz para la colisión de un nucleón incidente de energía $E'$ con un núcleo del aire, produciendo otro nucleón con energía $E$. Para generalizar al caso de la propagación de los múltiples hadrones producidos, se considera un grupo de ecuaciones acopladas \cite{Gaisser1990}:
	\begin{align}
	\dv{N_i(E,X)}{X} = -\qty(\frac{1}{\lambda_i}+\frac{1}{d_i}) N_i(E,X) + \sum_i \int \frac{F_{ij}(E_i,E_j)}{E_i} \frac{N_j(E_j)}{\lambda_j} \dd{E_j}.
	\end{align}

		\subsubsection*{Modelo Heitler-Matthews}
		En 1954, W. Heitler presentó un modelo simplificado del desarrollo de la componente electromagnética, posteriormente modificado por J. Matthews. Aunque no reemplaza simulaciones detalladas de cascadas, el modelo Heitler-Matthews permite describir correctamente características importantes de las mismas. En el modelo de Heitler se describe la componente electromagnética como ilustra la figura \ref{fig:heitler_em}; 
		\begin{wrapfigure}{l}{0.3\textwidth}
		\includegraphics[width=0.3\textwidth]{Figuras/heitler_em.png} 
		\caption{Esquema de un chubasco puramente electromagnético; las líneas rectas representan electrones y las curvas representan fotones.}
		\label{fig:heitler_em}
		\end{wrapfigure}		
		luego de viajar una distancia $d=\lambda_r \ln 2$ (donde $\lambda_r$ es la longitud de radiación en el aire) un electrón produce un fotón que al viajar la misma distancia produce un par $e^- e^+$. Luego de $n$ divisiones, en la cascada hay un total de $N=2^n$ partículas; la división se detiene cuando las partículas alcanzan una energía crítica $\xi^e_c$. \\
		
		A partir de lo anterior pueden deducirse características de una cascada iniciada por un fotón:
		\begin{align}
		E_0 &= \xi ^e_c N_{\text{max}} \label{e0_em}, \\
		X^{\gamma}_{\text{max}} &= n_c \lambda \ln 2 = \lambda \ln[E_0/\xi ^e_c] \label{xmax_em},
		\end{align}
		donde $n_c$ es el número de longitudes $d$ necesarias para que la energía por partícula se reduzca a $\xi_c^e$, donde $N = N_{\text{max}}= 2^{n_c}$. Se observa que el número de partículas en el máximo aumenta linealmente con la energía inicial y que la profundidad aumenta con la energía de manera logarítmica.
		\begin{wrapfigure}{r}{0.3\textwidth}
		\includegraphics[width=0.3\textwidth]{Figuras/heitler_h.png} 
		\caption{Esquema de un chubasco producido por un protón; las líneas sólidas representan piones cargados mientras las líneas cortadas representan piones neutros.}
		\label{fig:heitler_h}
		\end{wrapfigure}	 
		Los chubascos iniciados por hadrones se describen similarmente, como se ilustra en la figura \ref{fig:heitler_h}. Se consideran capas de atmósfera de altura $\lambda_I \ln 2$ donde $\lambda_I$ es la longitud de interacción. Luego de atravesar una capa, un hadrón interactúa produciendo partes iguales de piones cargados y neutros; $N_{\pi^{\pm}}$ cargados y $N_{\pi^{0}}=\frac{1}{2}N_{\pi^{\pm}}$ neutros. Los piones cargados pueden repetir el proceso hasta alcanzar una energía crítica $\xi_c^{\pi}$, entonces se asume su decaimiento. \\
		
		Luego de $n$ capas, la energía por pion cargado es
		\begin{align}
		E_{\pi} = \frac{E_0}{\qty(\frac{3}{2}N_{\pi^{\pm}})^n},
		\end{align}
		de manera que el número de interacciones $n_c$ necesarias para que la energía por pion alcance el valor crítico es
		\begin{align}
		n_c = \frac{\ln \qty[E_0 / \xi_c^{\pi}]}{\ln \qty[\frac{3}{2}N_{\pi^{\pm}}]}.
		\end{align}
		Considerando una cascada iniciada por un protón, con componentes electromagnética y hadrónica, y tomando en cuenta el decaimiento de todos los piones en muones ($(N_{\pi^{\pm}}^{n_c} = N_{\mu}$) la energía total es
		\begin{align} \label{totalEnergy}
		E_0 = \xi_c^e N_{\text{max}}+\xi_c^{\pi} N_{\mu}.
		\end{align}
		El modelo de Heitler sobreestima la razón de electrones a fotones, de manera que se introduce un factor de corrección $g=10$ tal que $N_e = N/g$. Con esta corrección la ecuación (\ref{totalEnergy}) se reescribe como
		\begin{align} \label{e0_h}
		E_0 = g \xi_{c}^{e} \qty(N_e + \frac{\xi_c^{\pi}}{g \xi_c^{e}} N_{\mu}),
		\end{align}
		lo que indica que la energía inicial puede calcularse si se miden el número de electrones y de muones. Cabe mencionar que esta expresión es independiente del tipo de partícula primaria. Asimismo se puede estimar la profundidad del máximo $X_{\text{max}}^{p}$ tomando en cuenta que el protón primario interactúa a una profundidad $X_0=\lambda_I \ln 2$ como
		\begin{align} \label{xmax_h}
		X_{\text{max}}^{p} &= X_0 + \lambda_r \ln \qty[\frac{E_0}{3 N_{\pi^{\pm}}\xi_c^e}].
		\end{align}
		El segundo término corresponde a la profundidad del máximo de la componente electromagnética de la ecuación (\ref{xmax_em}), cuya energía inicial es $E_0/3N_{\pi^{\pm}}$. En las ecuaciones (\ref{e0_h}) y (\ref{xmax_h}) se observa que la dependencia del número de partículas y la profundidad del máximo con la energía inicial es líneal y logarítmica, respectivamente, tal como en (\ref{e0_em}) y (\ref{xmax_em}). En general los valores calculados con la ecuación (\ref{xmax_h}) son bastante bajos comparados con resultados de simulaciones; esto debido a que sólo se toma en cuenta la primera generación de partículas de la componente electromagnética. Por otro lado, el número de muones en la cascada puede expresarse en función de la energía como
		\begin{align}
		N_{\mu} &= \qty(\frac{E_0}{\xi_c^{\pi}})^{\beta}, \text{ donde } \beta = \frac{\ln[N_{\pi^{\pm}}]}{\ln[\frac{3}{2}N_{\pi^{\pm}}]}.
		\end{align}

	Para describir chubascos producidos por un núcleo $A$, en el modelo Heitler-Matthews se utiliza el modelo de superposición; se trata el núcleo como $A$ protones iniciando cascadas individualmente, cada uno con una porción igual de la energía inicial del núcleo $E_0$, es decir $E_0/A$. Las características de estas cascadas pueden obtenerse sustituyendo la energía inicial en las ecuaciones para protones, a la vez que se expresan en términos de las cantidades correspondientes a un chubasco producido por un protón de energía $E_0$, por ejemplo el número de muones y la profundidad del máximo puede expresarse como
	\begin{align}
	N_{\mu}^{A}		  	&= N_{\mu}^{p} A^{\beta -1}, \label{eq: NmuA} \\ 
	X_{\text{max}}^{A} 	&= X_{\text{max}}^{p} - \lambda_r \ln A. \label{eq: XmaxA}
	\end{align}

	Adicionalmente, Matthews presenta el modelo tomando en cuenta la inelasticidad de las interacciones; como resultado de una interacción se produce una partícula principal que se lleva la mayor parte de la energía, de manera que hay menos energía disponible para la producción de las múltiples partículas restantes. La inelasticidad se introduce con un parámetro $\kappa$ representando la porción de la energía inicial que se invierte en la producción de piones. Todas las expresiones anteriores corresponden a un valor $\kappa = 1$.

	
	\subsection{Métodos de observación}
	Se han diseñado experimentos con distintos principios para la observaci\'on de cascadas; \'estos pueden ser de \textit{radiación de Cherenkov}, como el \textit{High-Altitude Water Cherenkov Gamma-Ray Observatory} (HAWC) que detectan radiación producida por una partícula cargada que se mueve a través de un medio más rápido que la velocidad de la luz en ese medio; y de \textit{fluorescencia}, que colectan la luz emitida por las moléculas de nitrógeno excitadas en el chubasco, este método permite reconstruir el desarrollo longitudinal del mismo. Existen también observatorios híbridos, como el \textit{Telescope Array} (TA) y el \textit{Pierre Auger Observatory} (PAO); estos utilizan la técnica de fluorescencia para observar el desarrollo de la cascada y además detectan partículas de alta energía que alcanzan el nivel del suelo.
	
\section{Estado del conocimiento}
%Outline:
%- Preguntas abiertas en la fisica de rayos cosmicos

En la f\'isica de rayos c\'osmicos las preguntas fundamentales est\'an relacionadas con su origen y el mecanismo por el cu\'al adquieren su energ\'ia. Para energ\'ias cercanas a los TeV, los rayos c\'osmicos pueden detectarse directamente desde observatorios espaciales, sin embargo a medida que la energ\'ia aumenta y su flujo disminuye es necesaria la detecci\'on indirecta por medio de la observaci\'on de cascadas atmosf\'ericas desde el suelo. A partir de estas observaciones se recopilan las caracter\'isticas de los rayos c\'osmicos que pueden luego confrontarse con modelos te\'oricos para inferir informaci\'on sobre su origen. \\

%- Rayos cosmicos en el rango de los TeV
En la actualidad, los RC de ultraalta energía (UHECR, $E > 10^{18}$ eV) siguen considerándose un enigma en términos de su composición y origen. Experimentos como TA y PAO miden observables de las cascadas, entre ellas $X_{max}$, $X_{max}^{\mu}$ y $N_{\mu}$, que son sensibles a la energía y la masa de la partícula primaria y por tanto pueden aportar al espectro de RC y dar información sobre la composicion del flujo de los UHECR. Con ayuda de programas de simulación de chubascos atmosféricos, como CORSIKA y AIRES, se ha progresado en esta dirección. \\

Dichos programas utilizan modelos de interacciones hadrónicas, que son mayoritariamente fenomenológicos, consecuentemente el estudio de los UHECR está estrechamente vinculado con la investigación experimental de colisiones hadrónicas de alta energía en aceleradores de partículas. Los parámetros de las interacciones que afectan el desarrollo de los chubascos atmosféricos son la sección eficaz, la multiplicidad y la elasticidad; se han comparado diferentes modelos (Sibyll 2.3, EPOS LHC y QGSJETII-04) observando que coinciden muy bien para interacciones $p-p$, pero difieren para interacciones $p-A$ y $\pi-A$ \cite{Pierog2018}. \\

\begin{figure}[h]
\centering
\includegraphics[width=0.95\textwidth]{Figuras/Pierog2018} 
\caption{Sección eficaz inelástica para interacciones p-p (izquierda) e interacciones p-aire y $\pi$-aire (derecha) calculadas con tres modelos hadrónicos \cite{Pierog2018}.}
\label{fig:cross_sections}
\end{figure}	

Por otro lado, otras observables que pueden ser fuente de información sobre la composición primaria de los rayos cósmicos son las distribuciones laterales de partículas (densidad de partículas a nivel de suelo en función de la distancia radial a partir del eje de la cascada) o la densidad de partículas a cierta distancia. Usualmente la distancia a la que se estudia la señal o la densidad de partículas es propia de cada observatorio y se denomina \textit{distancia óptima} ($r_{\text{opt}}$) \cite{Newton2007}, que se define como la distancia a la cual se minimizan las fluctuaciones estadísticas del ajuste de los datos a una función de distribución lateral. Estas mediciones se utilizan principalmente para determinar la energía inicial y la posición del eje del chubasco. \\

Distribuciones laterales de diversas partículas han sido medidas por varios experimentos; el experimento KASCADE (\textit{KArlsruhe Shower Core and Array DEtector}) ha realizado medidas de distribuciones laterales de electrones, muones y hadrones \cite{Antoni2001}, encontrando que los datos pueden describirse por funciones de tipo NKG. Asimismo, el PAO ha contemplado una función de distribución lateral de la señal en eventos detectados por su arreglo superficial \cite{Barnhill2005}, concluyendo -luego de ajustar los datos a diferentes funciones- que una función de tipo NKG provee el mejor ajuste, y determinando $r_{\text{opt}} = 1000$ m para el observatorio. Más recientemente se han reportado resultados preliminares de AMIGA (\textit{Auger Muons and Infill for the Ground Array}), una extensión del PAO, donde se ha medido directamente la densidad de muones en cascadas \cite{Muller2019}, confirmando que los modelos de interacción hadrónica no reproducen correctamente los datos en el rango de las UHE. \\

Comparando datos experimentales con simulaciones se han interpretado datos de AGASA (\textit{Akeno Giant Air Shower Array}) de distribuciones laterales de electrones, fotones y muones; los resultados de simulaciones describen bien los datos, independientemente del modelo hadrónico y la composición primaria \cite{Nagano2000}. Tambi\'en se han comparado datos de KASCADE con simulaciones hechas con distintos modelos de interacción hadrónica y se mostró que las distribuciones de muones se reproducen bien, contrario a las de electrones que difieren de los datos experimentales \cite{Apel2005}, además se observó que ésta última sugiere un cambio a una composición pesada en el espectro de rayos cósmicos. \\

\begin{figure}[]
\centering
\includegraphics[width=0.6\textwidth]{Figuras/Apel2005} 
\caption{Comparación de distribuciones laterales de muones resultado de simulaciones con el programa CORSIKA y medidas del observatorio KASCADE \cite{Apel2005}.}
\label{fig:Apel}
\end{figure}


Igualmente se ha estudiado el impacto del modelo de interacciones hadrónicas en las funciones de distribución lateral a partir de datos experimentales. Por ejemplo, en \cite{Drescher2003} han comparado distintas combinaciones de modelos y concluyen que las funciones son dependientes de los modelos hadrónicos de bajas y altas energías. Asimismo, se ha explorado la relación entre la forma de las distribuciones laterales y la composición primaria de los UHECR; haciendo predicciones teóricas \cite{Raikin2001} y comparando con datos de AGASA se ha visto una alta dependencia de la función de distribución lateral con la energía y composición primarias, describiéndola con la variación del parámetro radio medio cuadrado $R_{m.s}$. \\

\begin{figure}[]
\centering
\includegraphics[width=0.5\textwidth]{Figuras/Raikin2001}
\caption{Radio medio cuadrado de electrones predicho teóricamente utilizando el formalismo de escala junto con modelos de interacción hadrónica utilizando el programa CORSIKA, mostrando su dependencia de la masa primaria \cite{Raikin2001}.}
\label{fig:Raikin}
\end{figure}

De manera que los modelos de interacci\'on hadr\'onica m\'as utilizados se han estudiado en t\'erminos de su impacto en ciertas caracter\'isticas de las cascadas atmosf\'ericas simuladas y su respectiva comparaci\'on con datos experimentales a energ\'ias $>10^{15}$ eV. No obstante a menores energ\'ias, en el rango de los TeV, tambi\'en se han reportado estudios de las diferencias entre modelos para algunas observables de las cascadas, donde se concluye que hay buena concordancia entre ellos para energ\'ias mayores que 10 TeV \cite{Parsons2019}. \\

El experimento HAWC es un ejemplo de observatorio a nivel de suelo que detecta part\'iculas producidas en cascadas atmosf\'ericas en el rango de los TeV utilizando sus 300 detectores de agua Cherenkov (WCDs por sus siglas en ingl\'es \textit{Water Cherenkov Detectors}) distribuidos en un \'area de $22000$ m$^2$ a una altura de 4100 m sobre el nivel del mar. Originalmente HAWC est\'a dise\~nado para estudiar rayos gamma, pero tambi\'en pueden estudiarse rayos c\'osmicos de hasta $10^{15}$ eV a trav\'es de las distribuciones laterales de las part\'iculas producidas en las cascadas.  Dichas distribuciones pueden contener informaci\'on sobre la part\'icula primaria como su energ\'ia y su masa. \\

Las observaciones de cascadas atmosf\'ericas en HAWC se han utilizado para medir diversas propiedades de los rayos c\'osmicos; se han confirmado y mejorado mediciones de la anisotrop\'ia en la direcci\'on incidente del flujo de los rayos c\'osmicos \cite{Abeysekara2018a}, el estudio de esta propiedad permite profundizar en la descripci\'on de la propagaci\'on de las part\'iculas a trav\'es de diferentes medios; tambi\'en se han extendido a mayores energ\'ias las mediciones del flujo de antiprotones \cite{Abeysekara2018} ayud\'andose del estudio del efecto lunar en el flujo de los rayos c\'osmico, imponiendo un l\'imite superior a la raz\'on $\bar{p}/p$ en el rango de 1-10 TeV; asimismo, se han observado potenciales fuentes de rayos c\'osmicos de origen gal\'actico además de remanentes de supernova \cite{Hooper2017, Malone2019, Abeysekara2021}.
%Un parrafo sobre el trabajo de HAWC sobre rayos cosmicos de TeV: mediciones de la anisotropia, medici[on del flujo de anti protones utilizando la sobra de la luna, Estudio de fuentes galacticas de rayos cosmicos.
\singlespace


