\spacing{1.25}
Las simulaciones de cascadas atmosf\'ericas producidas por rayos c\'osmicos son de gran importancia para poder interpretar mediciones experimentales en t\'erminos de las propiedades de la part\'icula primaria. Para observatorios de detectores superficiales de agua Cherenkov es de especial inter\'es la distribuci\'on lateral de las part\'iculas secundarias que llegan al suelo. El objetivo de este trabajo es estudiar la distribuci\'on lateral de la componente mu\'onica en cascadas de energ\'ia primaria entre 1 y 100 TeV, para ello se realizaron simulaciones con el programa AIRES de cascadas iniciadas por protones y n\'ucleos de hierro, cuyo eje se encuentre en la ubicaci\'on del observatorio HAWC. Se caracteriz\'o la distribuci\'on a partir de los par\'ametros del ajuste a una funci\'on de tipo NKG, los par\'ametros est\'an relacionados con el n\'umero de muones y la edad de la cascada. Difiriendo de las predicciones de modelos anal\'iticos, se observa que en las distancias de inter\'es las cascadas de protones presentan mayor n\'umero de muones que las de hierro hasta $\approx 30$ TeV, mientras que al incrementar la energ\'ia la situaci\'on se invierte. Adem\'as es notable que los par\'ametros de la distribuci\'on no son observables que permitan discriminar entre modelos de interacciones hadr\'onicas. Por \'ultimo se estima que en cascadas de rayos c\'osmicos de bajas energ\'ias no llegar\'ian suficientes muones para poder distinguirlas con HAWC.

\vfill
 
\singlespacing