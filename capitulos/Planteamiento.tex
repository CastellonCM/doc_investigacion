\spacing{1.25}
\section{Objetivos}
	\subsection*{Objetivo general}
	Estudiar el efecto de una composición primaria mixta de los UHECR en la distribución lateral de electrones y muones, con el fin de observar si la composición primaria que se ha obtenido por la colaboración Pierre Auger, a partir de mediciones de la profundidad de máximo y el número de muones, es consistente con otras observables de los chubascos.	

	\subsection*{Objetivos específicos}	
	\begin{enumerate}
	\item Verificar el efecto de una composición mixta de los UHECR en las observables $X_{\text{max}}$ y $\sigma X_{\text{max}}$ realizando simulaciones de chubascos atmosféricos con el sistema AIRES y comparando con resultados publicados por la colaboración Pierre Auger para validar dichas simulaciones.
	
	\item Analizar las distribuciones laterales de electrones y muones en chubascos producidos por rayos cósmicos de composición mixta, en contraposición con chubascos producidos únicamente por protones o núcleos de hierro.
	
	\item Examinar las discrepancias entre las predicciones de las distribuciones laterales realizadas con los diferentes modelos hadrónicos de altas energías disponibles en AIRES, con el propósito de concluir si el efecto de la composición es sensible a particularidades de las interacciones hadrónicas.
	
	\item Comparar los resultados de las distribuciones laterales obtenidas de las simulaciones con los respectivos datos tomados por el Observatorio Pierre Auger.
	\end{enumerate}

\section{Preguntas de investigación}
A partir de datos experimentales del PAO se ha intentado deducir la composición de los rayos cósmicos de las más altas energías, para ello se han realizado ajustes a observables de chubascos atmosféricos tales como la profundidad del máximo y sus fluctuaciones, y el número de muones a nivel del suelo, sin embargo los resultados no son concluyentes. En esta investigación se pretende analizar la relación de la composición con otras observables: la distribución lateral de electrones y la distribución lateral de muones. \\

Para ello se plamtean las siguientes interrogantes: ¿cómo afecta la composición del flujo de UHECR a la distribución lateral de electrones y de muones? ¿la distribución lateral de electrones y muones obtenida en simulaciones es sensible al modelo de interacciones hadrónicas utilizado? ¿los resultados de la distribución lateral apoyan la hipótesis de una composición primaria mixta? y ¿la composición mixta deducida a partir de otras observables es congruente con los resultados y mediciones de la distribución lateral?

\section{Justificación}
Los rayos cósmicos de ultraalta energía se estudian mediante observaciones y simulaciones de las cascadas de partículas que producen en la atmósfera. Actualmente existen discrepancias entre las simulaciones realizadas con diferentes modelos de interacciones hadrónicas, así como entre predicciones de simulaciones y resultados observacionales. Debido a tales desacuerdos no es posible hacer coincidir simulaciones y observaciones en términos de una única composición de los UHECR.\\

Las magnitudes de un chubasco atmosférico con mayor dependencia de la masa del rayo cósmico primario son la profundidad del máximo y sus fluctuaciones, la profundidad de la máxima producción de muones y el número de muones a nivel del suelo. Tomando en cuenta las mediciones de dichos observables, la colaboración Pierre Auger ha propuesto una composición primaria mixta de los rayos cósmicos. \\

Esta investigación pretende estudiar los efectos de la composición de los UHECR en diferentes observables del desarrollo de chubascos atmosféricos mediante simulaciones, utilizando distintos modelos hadrónicos de altas energías. De esta manera se logrará una mejor comprensión del comportamiento de las distribuciones laterales de electrones y muones en relación a la composición del flujo de rayos cósmicos primarios, de modo que al comparar las simulaciones con los datos observacionales pueda apoyarse o contradecirse la hipósisis de una composición mixta de los UHECR.\\

Asimismo, se tiene por objetivo analizar la sensibilidad de los resultados de las distribuciones laterales al modelo de interacción hadrónica. Dado que la tarea de reconciliar los resultados de las simulaciones ejecutadas con distintos modelos se ha mostrado difícil -debido a que las magnitudes más utilizadas presentan una gran dependencia del modelo-, con este análisis se busca poder sugerir el empleo de estos observables en nuevos estudios acerca de la composición primaria de rayos cósmicos.

\section{Viabilidad}
Para las simulaciones se utilizará el sistema AIRES, éste es de acceso libre y está disponible en línea en el sitio \url{http://aires.fisica.unlp.edu.ar/}, así como toda su documentación; éste ya ha sido instalado en una computadora personal junto con los modelos hadrónicos disponibles, adicionalmente se han realizado ejecuciones de prueba para estimar el tiempo de simulación. Se estima que por cada modelo se empleen tres semanas para todas las simulaciones necesarias, haciendo un total de nueve semanas, siendo éste tiempo razonable dado el tiempo disponible para el desarrollo del proyecto.\\ 

Por otro lado, muchos datos del Observatorio Pierre Auger están publicados en su repositorio en su sitio web, incluyendo los resultados de las fracciones de núcleos primarios que conforman la composición mixta. Cabe mencionar que la asesora principal del proyecto, PhD. Karen Caballero, es parte de la colaboración Pierre Auger así como de la colaboración del \textit{High Altitude Water Cherenkov Gamma-ray Observatory} (HAWC). Además es líder del grupo de investigación de astropartículas de la Facultad de Ciencias en Física y Matemáticas de la Universidad Nacional Autónoma de Chiapas. De manera que su experiencia en esta área de investigación es la oportuna para la realización del presente proyecto.



\singlespace
