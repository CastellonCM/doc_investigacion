\DIFdelbegin %DIFDELCMD < \spacing{1.25}
%DIFDELCMD < %%%
\DIFdelend \DIFaddbegin \spacing{1.25}
\vspace{-5mm}

\DIFaddend 

Los rayos cósmicos -descubiertos por el austríaco Victor Hess en 1914- son partículas cargadas provenientes del exterior de la Tierra que llegan a la misma con energías de hasta $10^{20}$ eV. En una primera aproximaci\'on el espectro de los rayos c\'osmicos est\'a compuesto por protones (90\%) y núcleos de helio (9\%), siendo el resto electrones, positrones y núcleos más pesados. Al ingresar a la atmósfera terrestre los rayos c\'osmicos interactúan con los átomos y moléculas que la conforman, generando m\'ultiples partículas secundarias cuyo conjunto se conoce como cascada atmosf\'erica, éstas son producto de interacciones electromagnéticas y hadrónicas. \\

Las simulaciones de cascadas atmosf\'ericas producidas por rayos c\'osmicos son esenciales para la interpretaci\'on de las mediciones de las mismas en observatorios. Ya que los rayos c\'osmicos no viajan en l\'inea recta desde su fuente hasta la Tierra, no es posible identificar directamente dichas fuentes, por lo que se estudia la anisotrop\'ia del flujo y la composici\'on (masa) en funci\'on de la energ\'ia de los rayos c\'osmicos para dilucidar su origen \cite{Albrecht2021}. A las m\'as altas energ\'ias las caracter\'isticas de los rayos c\'osmicos primarios deben inferirse a partir de observables que son afectadas por las fluctuaciones cascada a cascada. En el caso de la masa, una dificultad que se presenta es que las fluctuaciones muchas veces sobrepasan a las diferencias entre cascadas de distinta masa. \\

El n\'umero de muones $N_{\mu}$ y su distribuci\'on lateral en una cascada est\'a relacionado tanto con la energ\'ia como con la masa y \'angulo de incidencia del rayo c\'osmico; la incerteza asociada a $N_{\mu}$ es principalmente ocasionada por c\'omo se describe la evoluci\'on de la componente hadr\'onica de la cascada, dicha descripci\'on se hace a partir de modelos computacionales alimentados por datos de aceleradores de part\'iculas. Diferentes autores y colaboraciones han observado discrepancia entre la componente mu\'onica observada y la predicha por medio de simulaciones utilizando modelos de interacciones hadr\'onicas de altas energ\'ias\DIFdelbegin \DIFdel{. La }\DIFdelend \DIFaddbegin \DIFadd{, atribuy\'endola principalmente a la extrapolaci\'on a altas energ\'ias que realizan los modelos, sin embargo la }\DIFaddend imposibilidad de resolver la mencionada discrepancia \DIFaddbegin \DIFadd{modificando par\'ametros }\DIFaddend sugiere que en los modelos hace falta alg\'un efecto f\'isico que no se ha observado en los experimentos de aceleraci\'on de part\'iculas \DIFaddbegin \DIFadd{\cite{Albrecht2021}}\DIFaddend . \\

La distribuci\'on lateral de muones es de particular importancia para experimentos de detectores superficiales como el observatorio HAWC, principalmente para poder distinguir entre cascadas de rayos gamma y de rayos c\'osmicos. HAWC es un experimento conformado por 300 detectores de agua Cherenkov que puede observar cascadas atmosf\'ericas con energ\'ias en el orden de los TeV. Est\'a ubicado en Puebla, M\'exico, a una altura de 4100 m s.n.m. cubriendo un \'area de 22000 m$^2$, donde utilizando las distribuciones laterales de part\'iculas se reconstruyen las cascadas, obteniendo informaci\'on tanto de las interacciones de las part\'iculas secundarias en la atmosf\'era como de la naturaleza de las part\'iculas primarias.\\

El objetivo de este trabajo de investigaci\'on es estudiar las cascadas atmosf\'ericas a trav\'es de las distribuciones laterales de la componente mu\'onica de las mismas, que debido a que los muones tienen su origen principalmente en interacciones hadr\'onicas, esta componente est\'a estrechamente relacionada con las caracter\'isticas del rayo c\'osmico inicial. Particularmente, se caracteriza dicha distribuci\'on lateral en funci\'on de la energ\'ia primaria y se compararan las distribuciones en cascadas iniciadas por n\'ucleos de distinta masa y con distinto \'angulo de incidencia. Lo anterior se lleva a cabo mediante simulaciones de cascadas con diferentes modelos de interacciones hadrónicas, en este caso: Sibyll 2.3d \DIFaddbegin \DIFadd{\cite{Riehn2020}}\DIFaddend , EPOS-LHC \DIFaddbegin \DIFadd{\cite{Pierog2015} }\DIFaddend y QGSJETII-04 \DIFaddbegin \DIFadd{\cite{Ostapchenko2011}}\DIFaddend .\\

Utilizando el programa AIRES \DIFaddbegin \DIFadd{(v19.04.06) \cite{Sciutto2021} }\DIFaddend se realizaron seis grupos de simulaciones de aproximadamente 20,000 eventos de cascadas atmosf\'ericas cada uno; por una parte se simularon cascadas producidas por protones, y por otra, cascadas iniciadas por n\'ucleos de hierro, todas ellas haciendo uso de los tres modelos hadr\'onicos separadamente, a manera de poder contrastar los resultados de cada uno. La energ\'ia inicial de las cascadas est\'a en el rango de 1-100 TeV, el \'angulo azimutal de incidencia es isotr\'opico, mientras que el zenital est\'a entre 0 y $45^{\circ}$ y la ubicaci\'on es la del observatorio HAWC. A partir de las simulaciones se caracteriz\'o la distribuci\'on lateral de muones utilizando los par\'ametros del ajuste a una funci\'on de tipo NKG, que est\'an relacionados con el n\'umero de muones y la edad de la cascada.\\

A bajas energ\'ias se observa que en las distancias de inter\'es hay una aparente discrepancia con las predicciones del modelo Heitler-Matthews \cite{Matthews2005}: las cascadas de protones presentan mayor n\'umero de muones que las de hierro hasta $\approx 30$ TeV, mientras que al incrementar la energ\'ia la situaci\'on se invierte. Por otra parte, al comparar los resultados de los tres modelos de interacciones hadr\'onicas es notable que los par\'ametros de la distribuci\'on no son observables que permitan discriminar entre dichos modelos para sugerir cu\'al de ellos realiza mejores predicciones. Por \'ultimo se estima que en cascadas de rayos c\'osmicos de bajas energ\'ias no llegar\'ian suficientes muones para poder distinguirlas de cascadas electromagn\'eticas.\\

El contenido del documento est\'a organizado de la siguiente manera: en el capítulo 1 se resume el fundamento teórico necesario para la comprensión física de las cascadas atmosfericas producidas por rayos cósmicos y sus principales propiedades, así como el estado del conocimiento del tema; en el capítulo 2 se describe el programa AIRES con el que se ejecutar\'an las simulaciones, así como las condiciones que se impondr\'an para las mismas; los resultados del trabajo de investigaci\'on y respectiva discusi\'on y an\'alisis se presenta en el cap\'itulo 3; y por \'ultimo, en el cap\'itulo 4 se exponen las conclusiones, adem\'as de proponer mejoras para un trabajo a futuro.

\singlespacing