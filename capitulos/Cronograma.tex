Se presenta el cronograma de actividades a seguir. A continuación se describen las tareas a realizar en el desarrollo de la investigación:
\begin{center}
\begin{footnotesize}
\begingroup
\setlength{\tabcolsep}{8pt} % Default value: 6pt
\renewcommand{\arraystretch}{1.3} % Default value: 1
\begin{tabular}{|c|c|c|c|c|c|c|c|c|c|c|c|c|c|c|c|c|c|c|c|c|}
\hline
\textbf{Semana} & \textbf{01} & \textbf{02} & \textbf{03} & \textbf{04} & \textbf{05} & \textbf{06} & \textbf{07} & \textbf{08} & \textbf{09} & \textbf{10} & \textbf{11} & \textbf{12} & \textbf{13} & \textbf{14} & \textbf{15} & \textbf{16} & \textbf{17} & \textbf{18} & \textbf{19} & \textbf{20}\\ \hline
Tarea 1 & \cellcolor{gray}	& \cellcolor{gray}	& \cellcolor{gray}	& \cellcolor{gray}	& \cellcolor{gray}	& \cellcolor{gray}	& \cellcolor{gray}	& \cellcolor{gray}	& \cellcolor{gray}	& \cellcolor{gray}	& \cellcolor{gray}	& \cellcolor{gray}	& \cellcolor{gray}	& \cellcolor{gray}	& \cellcolor{gray}	& 	&	&	&	& 							  	  \\ \hline
Tarea 2 & \cellcolor{gray}	& 	& 	& 	&	&	&	&	&	&	&	&	&	&	&	&	&	&	&	& 	  \\ \hline
Tarea 3 &	& \cellcolor{gray}	& \cellcolor{gray}	& \cellcolor{gray}	& 	&	&	&	&	&	&	&	&	&	&	&	&	&	&	& \\ \hline
Tarea 4 &	&	&	& 	& \cellcolor{gray}	& \cellcolor{gray} 	&	&	&	&	&	&	&	&	&	&	&	&	&	& \\ \hline
Tarea 5 &	&	&	& 	& \cellcolor{gray}	& \cellcolor{gray}	& \cellcolor{gray}	&	&	&	&	&	&	&	&	&	&	&	&	& \\ \hline
Tarea 6 &	&	&	&	&	&	&	& \cellcolor{gray}	& \cellcolor{gray}	&	&	&	&	&	&	&	&	&	&	& \\ \hline
Tarea 7 &	&	&	&	&	& 	& 	& \cellcolor{gray}	& \cellcolor{gray}	& \cellcolor{gray}	&	&	&	&	&	&	&	&	&	& \\ \hline
Tarea 8 &	&	&	&	&	&	&	& 	& 	& 	& \cellcolor{gray}	&	\cellcolor{gray} &	&	&	&	&	&	&	& \\ \hline
Tarea 9 & 	&	&	&	&	&	&	&	&	&	& 	&  	& \cellcolor{gray}	& \cellcolor{gray}	& \cellcolor{gray}	&	&	&	&	& \\ \hline
Tarea 10 & 	&	&	&	&	&	&	&	&	&	&	& 	&	& 	& 	& \cellcolor{gray}	& \cellcolor{gray}	& \cellcolor{gray}	& \cellcolor{gray}	& \cellcolor{gray} \\ \hline
\end{tabular}
\endgroup
\end{footnotesize}
\end{center}

\begin{itemize}
\item \textbf{Tarea 01}: revisión de bibliografía.
	
\item \textbf{Tarea 02}: creación de archivos de entrada para cada grupo de simulaciones.

\item \textbf{Tarea 03}: simulación de chubascos producidos por protones y hierro con el modelo Sibyll 2.3d.

\item \textbf{Tarea 04}: análisis de resultados de distribuci\'on lateral de muones.

\item \textbf{Tarea 05}: simulación de chubascos producidos por protones y hierro con el modelo EPOS-LHC.

\item \textbf{Tarea 06}: análisis de resultados de distribuci\'on lateral de muones.

\item \textbf{Tarea 07}: simulación de chubascos producidos por protones y hierro con el modelo QGSJETII-04.

\item \textbf{Tarea 08}: análisis de resultados de distribuci\'on lateral de muones.

\item \textbf{Tarea 09}: comparación entre resultados de diferentes modelos.

\item \textbf{Tarea 10}: redacción de documento final.
\end{itemize}



